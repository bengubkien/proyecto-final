\chapter*{Conclusiones}
\label{conclusiones}
\addcontentsline{toc}{chapter}{Conclusiones}

En el presente trabajo se realizó un estudio de sistemas híbridos de almacenamiento basados en baterías de litio y supercapacitores orientados a aplicaciones móviles. En este sentido, se trabajó en el modelado, simulación, diseño e implementación de controladores para convertidores electrónicos de potencia, ensayando distintas topologías.

Particularmente, se diseñaron lazos de control de corriente y tensión, aplicados sobre topologías de uno y dos convertidores CC-CC elevadores bidireccionales en corriente. Los controladores fueron implementados en arquitectura FPGA mediante el lenguaje de programación VHDL. Los controladores fueron ensayados en una plataforma híbrida con un banco de supercapacitores y una fuente de potencia, emulando un banco de baterías. Se realizaron pruebas tanto en forma individual como conjunta, analizando su comportamiento y realizando ajustes finos sobre los parámetros.

En el caso de los ensayos con un único convertidor, se realizó un lazo interno de control de corriente y un lazo externo de control de tensión de bus, para tres casos diferentes. Dos de ellos con el banco de supercapacitores conectado del lado de baja tensión, en paralelo con la fuente principal. En un primer caso conectado directamente y en un segundo caso intercalando un inductor, de forma tal de incorporar una dinámica entre las respuestas de cada uno. El tercer caso consistió en la conexión del banco de supercapacitores en el lado de alta tensión, en paralelo con la carga. En todos los casos, se analizó la respuesta dinámica general del sistema, como también las respuestas individuales de la fuente de potencia y los supercapacitores ante cambios de referencia y de carga. En todos estos ensayos se obtuvieron excelentes resultados, permitiendo con una topología relativamente sencilla alimentar una misma carga desde dos fuentes de energía, repartiendo en cierta medida la velocidad de respuesta de cada una. En este sentido, se determinó que si bien es una estructura práctica y simple de implementar, no brinda grandes posibilidades de controlar o modificar la respuesta individual de cada fuente, dada la falta de grados de libertad del sistema.

Como ultimo caso de estudio, se ensayó una topología basada en dos convertidores CC-CC, uno asociado a cada fuente de energía y conectados a un mismo bus de continua. En este caso se implementaron dos controles de corriente (uno para cada convertidor) y un lazo de tensión de bus. Además, se implementó un lazo de control de tensión de los supercapacitores, necesario para mantenerlos a un nivel de carga adecuado. Esta topología demostró mucha mayor versatilidad a la hora de controlar la respuesta dinámica de cada fuente, asegurando la entrega de potencia a la carga en todo momento. El hecho de contar con dos convertidores, también aporta a la confiabilidad de sistema, teniendo dos caminos para alimentar a la carga. 

Durante el transcurso del proyecto, un obstáculo en particular resultó en la necesidad de comprender con mayor profundidad el mecanismo de discretización y representación de tipos de datos para la implementación en el FPGA. Este problema es detallado en el Apéndice \ref{apendice-error}.

En conclusión, se considera que el objetivo principal de este proyecto fue cumplido exitosamente. Como trabajo futuro en el marco de este proyecto, el sistema híbrido puede mejorarse y ampliarse a través de la mejora de la instrumentación de los convertidores electrónicos de potencia CC-CC, la evaluación de otras técnicas de control para observar si se presenta una mejor performance, o incluso la adición de nuevos módulos de almacenamiento de energía.

\newpage