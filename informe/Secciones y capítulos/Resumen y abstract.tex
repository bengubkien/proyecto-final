\newenvironment{abstractpage}
  {\cleardoublepage\vspace*{\fill}\thispagestyle{empty}}
  {\vfill\cleardoublepage}
\renewenvironment{abstract}[1]
  {\bigskip%
   \begin{center}\bfseries\abstractname\end{center}}
  {\par\bigskip}

\begin{abstractpage}
\begin{abstract}
    .Un sistema híbrido de almacenamiento de energía consiste en dos o más tipos de tecnologías de almacenamiento de energía, usualmente incluyendo baterías, supercapacitores, y pilas de combustible. Las características complementarias de estos sistemas híbridos hacen que superen a cualquier dispositivo de almacenamiento de energía individual, dependiendo de los requerimientos energéticos de la aplicación en distintos escenarios o bajo ciertas condiciones. Para resolver las limitaciones opuestas de las baterías y supercapacitores (la baterías posee una alta densidad energética pero baja densidad de potencia, mientras que los supercapacitores poseen una baja densidad energética pero una alta densidad de potencia) un sistema híbrido compuesto por estos dos dispositivos y un convertidor electrónico CC-CC es propuesto. El propósito de este trabajo es la combinación de ambos elementos junto a la implementación de un sistema de control de tensión individual a cada dispositivo. 
\end{abstract}

\renewcommand{\abstractname}{Abstract}

\begin{abstract}
    A Hybrid Energy Storage System (HESS) consist of two or more types of energy storage technologies, mostly including batteries, supercapacitors, and fuel cells. The complementary features of HESS make it outperform any single energy storage device depending on the application energy requirementes in different scenarios or conditions. To overcome the opposing limitations of batteries and supercapacitors (the battery has relatively high energy density but low power density, as compared to the supercapacitor), a battery-supercapacitor HESS with a DC-DC converter is proposed. The purpose of this work is the combination  of both elements along with the implementation of a voltage control system for each one of the devices.
\end{abstract}
\end{abstractpage}