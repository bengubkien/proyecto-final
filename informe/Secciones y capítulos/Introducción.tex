\chapter*{Introducción}
\addcontentsline{toc}{chapter}{Introducción}

La industria automotriz se ha convertido en una de las industrias más importantes a nivel mundial, no solo económicamente, pero también por su impacto en el sector de investigación y desarrollo. Cada vez más elementos tecnológicos son introducidos en vehículos con el fin de mejorar la seguridad de los pasajeros y peatones. Además, hay una cantidad cada vez mayor de vehículos en las calles, lo que nos permite movernos de forma rápida y cómoda. Sin embargo, esto ha generado un aumento dramático de los niveles de contaminación en el aire en zonas urbanas (por ejemplo, de partículas finas, óxidos de nitrógeno, monóxido de carbono, dióxido de azufre, etc.).

Asimismo, y según un reporte de la Unión Europea, el sector de transporte es responsable de casi un 24\% del total de emisiones de dióxido de carbono (CO$_2$), mientras que el subsector de transporte por calle comprende un 74\% de ese porcentaje. Por lo tanto, las autoridades de las naciones más desarrolladas están alentando el uso de vehículos eléctricos (EVs, del inglés \emph{Electric Vehicles}) para disminuir la concentración de contaminantes en el aire, CO$_2$, así como otros gases de efecto invernadero. Los EVs ofrecen las siguientes ventajas sobre los vehículos tradicionales:

\begin{itemize}
    \item Cero emisiones: Este tipo de vehículos no emiten contaminantes de tubo de escape, CO$_2$, ni dióxido de nitrógeno (NO$_2$). Además, los procesos de manufacturación tienden a ser más cuidadosos con su impacto al medio ambiente, aunque la fabricación de baterías afecta negativamente a su huella de carbono.
    \item Simpleza: El número de elementos que compone al motor de un vehículo eléctrico es menor, lo que se traduce a un mantenimiento más barato. Los motores son más pequeños y compactos, no necesitan refrigración, y no son necesarios elementos que reduzcan el ruido generado por este. Este tipo de motores no sufren del desgaste causado por las combustiones internas, vibraciones, o corrosiones debido al combustible provocadas en un motor tradicional.
    \item Coste: El costo de mantenimiento del vehículo y el costo de la cantidad de electricidad requerida es mucho menor en comparación con los vehículos con motor de combustión interna. El coste de energía por kilómetro es mucho más bajo en EVs que en vehículos tradicionales.
    \item Eficiencia: Los vehículos eléctricos convierten hasta más del 77\% de la energía eléctrica de la red en potencia en las ruedas. Los vehículos convencionales son capaces de sólamente aprovechar un 12\% a un 30\% de la energía almacenada en su combustible. \cite{evefficiency}
\end{itemize}

Estos vehículos eléctricos deben poseer un sistema de almacenamiento de energía con alta densidad de energía másica para permitir una distancia de manejo larga, y alta densidad de potencia másica para la aceleración, frenado y manejo en ascenso. Sin embargo, las baterías actuales no pueden satisfacer ambos requerimientos al mismo tiempo. Por lo tanto, es necesario crear un sistema que reúna dos o más dispositivos cuyas características cumplan al menos una necesidad de los EVs. \cite{hessinev}

Combinar componentes en sistemas híbridos para aprovechar los beneficios de cada parte siempre fue una perspectiva atractiva. En los últimos años, varios proyectos han sido exitosos en construir estos sistemas híbridos de almacenamiento de energía para energía solar y eólica.

Aunque la idea no es nueva, la tecnología aún se encuentra en una fase temprana. Los sistemas híbridos de almacenamiento de energía (siglas HESS, del inglés \emph{hybrid energy storage systems}) pueden referirse a distintos tipos de arreglos, con lo único en común siendo que dos o más tipos de almacenamiento de energía son combinados para formar un único sistema.

No existe una única solución ideal de almacenamiento de energía para cada aplicación existente, ya que los diseños en el mercado típicamente se dividen en dos: para aplicaciones de potencia (gran entrega de energía en cortas ventanas de tiempo), o para aplicaciones con gran densidad energética (baja y constante entrega de energía durante grandes períodos de tiempo).

Los HESS típicamente combinan estas soluciones para cumplir aplicaciones que requieren tanto una rápida respuesta energética, como una alimentación constante de ella. Estos sistemas híbridos pueden compartir la misma elecrónica de potencia y hardware de conexión a la carga, reduciendo los costos iniciales y de mantenimiento.

En este contexto, este trabajo tiene por objetivo realizar un estudio de un tipo de sistema híbrido particular, en el cual se combinan baterías de litio y supercapacitores para aplicaciones móviles. En los primeros tres capítulos se brinda una introducción a los elementos fundamentales en los que se basa este trabajo: los sistemas de almacenamiento, los convertidores electrónicos CC-CC utilizados para poder crear el sistema híbrido, y finalmente la arquitectura digital utilizada para implementar el sistema de control. Luego, a partir de los requerimientos dados para el sistema, se diseña e implementa un sistema de control propuesto. Por último se realizan una serie de ensayos para corroborar el correcto funcionamiento del sistema a lazo cerrado de control. 

\newpage