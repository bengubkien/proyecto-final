\chapter*{Introducción}
\addcontentsline{toc}{chapter}{Introducción}

Combinar componentes en sistemas híbridos para aprovechar los beneficios de cada parte siempre fue una perspectiva atractiva. En los últimos años, varios proyectos han sido exitosos en construir estos sistemas híbridos de almacenamiento de energía para energía solar y eólica.

Aunque la idea no es nueva, la tecnología aún se encuentra en una fase temprana. Los sistemas híbridos de almacenamiento de energía (siglas HESS, del inglés \emph{hybrid energy storage systems}) pueden referirse a distintos tipos de arreglos, con lo único en común siendo que dos o más tipos de almacenamiento de energía son combinados para formar un único sistema.

No existe una única solución ideal de almacenamiento de energía para cada aplicación existente, ya que los diseños en el mercado típicamente se dividen en dos: para aplicaciones de potencia (gran entrega de energía en cortas ventanas de tiempo), o para aplicaciones con gran densidad energética (baja y constante entrega de energía durante grandes períodos de tiempo).

Los HESS típicamente combinan estas soluciones para cumplir aplicaciones que requieren tanto una rápida respuesta energética, como una alimentación constante de ella. Estos sistemas híbridos pueden compartir la misma elecrónica de potencia y hardware de conexión a la carga, reduciendo los costos iniciales y de mantenimiento.

En este contexto, este trabajo tiene por objetivo realizar un estudio de un tipo de sistema híbrido particular, en el cual se combinan baterías de litio y supercapacitores para aplicaciones móviles. En los primeros tres capítulos se brinda una introducción a los elementos fundamentales en los que se basa este trabajo: los sistemas de almacenamiento, los convertidores electrónicos CC-CC utilizados para poder crear el sistema híbrido, y finalmente la arquitectura digital utilizada para implementar el sistema de control. Luego, a partir de los requerimientos dados para el sistema, se diseña e implementa un sistema de control propuesto. Por último se realizan una serie de ensayos para corroborar el correcto funcionamiento del sistema a lazo cerrado de control. 

\newpage